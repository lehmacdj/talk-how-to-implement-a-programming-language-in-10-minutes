\documentclass[pdf,t]{beamer}

\usepackage{listings}
\usepackage{color}
\usepackage{amsmath}
\usepackage{qtree}
\usepackage{multicol}

\renewcommand\lstlistingname{Quelltext} % Change language of section name
\lstset{% General setup for the package
  language=Perl,
  basicstyle=\small\sffamily,
  numbers=left,
  numberstyle=\tiny,
  frame=tb,
  tabsize=2,
  columns=fixed,
  showstringspaces=false,
  showtabs=false,
  keepspaces,
  commentstyle=\color{red},
  keywordstyle=\color{blue}
}

\mode<presentation>{}

% The audience for this talk was a group of EIC software engineers/PMs at
% Microsoft in P3 an EIC group within identity.
\title{How to Implement a PL in $< 10 \text{min}$}
\subtitle{Programming languages are cool!}
\author{Devin Lehmacher}

\begin{document}

\begin{frame}
  \titlepage{}
\end{frame}

\begin{frame}{Background}
  \begin{itemize}
    \item In addition to studying subjects like algorithms, databases,
      operating systems, etc. in college
    \item I spent a lot of time studying Programming Language theory
      \begin{itemize}
        \item What makes a good programming language?
        \item How to implement programming languages?
      \end{itemize}
    \item Language design opens possibilities
    \item Helps you evaluate language features
  \end{itemize}
\end{frame}

\begin{frame}[fragile]{What Language?}
  \begin{itemize}
    \item<1-> Hint: we're not going to implement C\# in 10 minutes.
    \item<2-> IMP (short for imperative), a very simple programming language
\begin{verbatim}
# Compute nth triangle number
n := 5;
i := n;
result := 0;
while i > 0 do (
  result := result + i;
  i := i - 1
);
print n
\end{verbatim}
  \end{itemize}
\end{frame}

\begin{frame}{Basics}
  \begin{itemize}
    \item<1->
      For a typical interpreted programming language like Python or Javascript we have:
      \begin{align*}
        \text{Source Code} \xrightarrow{\text{Parser}} \text{AST} \xrightarrow{\text{Interpreter}} \text{Program Output}
      \end{align*}
    \item<2-> Only going to implement an interpreter; so we'll start with Abstract Syntax Trees (ASTs)
    \item<2-> An expression like $1 + 2 * (3 + 4)$ would be represented as this AST:\\
      \Tree [.{+} 1 [.{*} 2 [.{+} 3 4 ] ] ]
    % at this point will want to overview Haskell + show the implementation of
    % the calculator
  \end{itemize}
\end{frame}

\begin{frame}[fragile]{Demo Outline}
  \begin{itemize}
    \item
      \begin{multicols}{2}
        Printing + Basic command stuff
\begin{verbatim}
print 1;
print (2 + 3)
\end{verbatim}
      \columnbreak{}
        \Tree [.{Seq} [.{Print} 1 ] [.{Print} [.{+} 2 3 ] ] ]
      \end{multicols}
    \item Assignment
    \item
      \begin{multicols}{2}
        Conditionals (e.g.\ if)
\begin{verbatim}
if 1
then print 42
else print 0
\end{verbatim}
      \columnbreak{}
        \Tree [.{IfNez} 1 [.{Print} 42 ] [.{Print} 0 ] ]
      \end{multicols}
    \item While
  \end{itemize}
\end{frame}

\begin{frame}{That's all!}
  \begin{itemize}
    \item<1-> A more complete implementation of this simple language including a
      parser + repl can be found at https://github.com/lehmacdj/imp-lang. It is
      a little bit ``better'' of an implementation too:
      \begin{itemize}
        \item It separates out booleans as a separate type from integers
        \item It supports a fuller range of operations (e.g.\ all comparison
          operators and boolean operators; more arithmetic operations)
        \item It is purer (it doesn't use IORef + IO in the evaluator)
      \end{itemize}
    \item<2-> Questions? (hopefully I didn't go overtime)
  \end{itemize}
\end{frame}


\end{document}
